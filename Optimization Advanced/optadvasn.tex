\documentclass[journal,12pt,twocolumn]{IEEEtran}

\usepackage[utf8]{inputenc}
\usepackage{kvmap}
\usepackage{graphics} 

\usepackage{setspace}
\usepackage{gensymb}

\singlespacing

\usepackage{amsthm}

\usepackage{mathrsfs}
\usepackage{txfonts}
\usepackage{stfloats}
\usepackage{bm}
\usepackage{cite}
\usepackage{cases}
\usepackage{subfig}

\usepackage{longtable}
\usepackage{multirow}

\usepackage{enumitem}
\usepackage{mathtools}
\usepackage{steinmetz}
\usepackage{tikz}
\usepackage{circuitikz}
\usepackage{verbatim}
\usepackage{tfrupee}
\usepackage[breaklinks=true]{hyperref}
\usepackage{graphicx}
\usepackage{tkz-euclide}
\usepackage{float}

\usetikzlibrary{calc,math}
\usepackage{listings}
    \usepackage{color}                                            %%
    \usepackage{array}                                            %%
    \usepackage{longtable}                                        %%
    \usepackage{calc}                                             %%
    \usepackage{multirow}                                         %%
    \usepackage{hhline}                                           %%
    \usepackage{ifthen}                                           %%
    \usepackage{lscape}     
\usepackage{multicol}
\usepackage{chngcntr}

\DeclareMathOperator*{\Res}{Res}

\renewcommand\thesection{\arabic{section}}
\renewcommand\thesubsection{\thesection.\arabic{subsection}}
\renewcommand\thesubsubsection{\thesubsection.\arabic{subsubsection}}

\renewcommand\thesectiondis{\arabic{section}}
\renewcommand\thesubsectiondis{\thesectiondis.\arabic{subsection}}
\renewcommand\thesubsubsectiondis{\thesubsectiondis.\arabic{subsubsection}}

\hyphenation{op-tical net-works semi-conduc-tor}
\def\inputGnumericTable{}                                 %%

\lstset{
%language=C,
frame=single, 
breaklines=true,
columns=fullflexible
}
\begin{document}

\newtheorem{theorem}{Theorem}[section]
\newtheorem{problem}{Problem}
\newtheorem{proposition}{Proposition}[section]
\newtheorem{lemma}{Lemma}[section]
\newtheorem{corollary}[theorem]{Corollary}
\newtheorem{example}{Example}[section]
\newtheorem{definition}[problem]{Definition}

\newcommand{\BEQA}{\begin{eqnarray}}
\newcommand{\EEQA}{\end{eqnarray}}
\newcommand{\define}{\stackrel{\triangle}{=}}
\newcommand\hlight[1]{\tikz[overlay, remember picture,baseline=-\the\dimexpr\fontdimen22\textfont2\relax]\node[rectangle,fill=blue!50,rounded corners,fill opacity = 0.2,draw,thick,text opacity =1] {$#1$};}
\bibliographystyle{IEEEtran}
\providecommand{\mbf}{\mathbf}
\providecommand{\pr}[1]{\ensuremath{\Pr\left(#1\right)}}
\providecommand{\qfunc}[1]{\ensuremath{Q\left(#1\right)}}
\providecommand{\sbrak}[1]{\ensuremath{{}\left[#1\right]}}
\providecommand{\lsbrak}[1]{\ensuremath{{}\left[#1\right.}}
\providecommand{\rsbrak}[1]{\ensuremath{{}\left.#1\right]}}
\providecommand{\brak}[1]{\ensuremath{\left(#1\right)}}
\providecommand{\lbrak}[1]{\ensuremath{\left(#1\right.}}
\providecommand{\rbrak}[1]{\ensuremath{\left.#1\right)}}
\providecommand{\cbrak}[1]{\ensuremath{\left\{#1\right\}}}
\providecommand{\lcbrak}[1]{\ensuremath{\left\{#1\right.}}
\providecommand{\rcbrak}[1]{\ensuremath{\left.#1\right\}}}
\theoremstyle{remark}
\newtheorem{rem}{Remark}
\newcommand{\sgn}{\mathop{\mathrm{sgn}}}
\providecommand{\abs}[1]{\left\vert#1\right\vert}
\providecommand{\res}[1]{\Res\displaylimits_{#1}} 
\providecommand{\norm}[1]{$\left\lVert#1\right\rVert$}
%\providecommand{\norm}[1]{\lVert#1\rVert}
\providecommand{\mtx}[1]{\mathbf{#1}}
\providecommand{\mean}[1]{E\left[ #1 \right]}
\providecommand{\fourier}{\overset{\mathcal{F}}{ \rightleftharpoons}}
%\providecommand{\hilbert}{\overset{\mathcal{H}}{ \rightleftharpoons}}
\providecommand{\system}{\overset{\mathcal{H}}{ \longleftrightarrow}}
	%\newcommand{\solution}[2]{\textbf{Solution:}{#1}}
\newcommand{\solution}{\noindent \textbf{Solution: }}
\newcommand{\cosec}{\,\text{cosec}\,}
\providecommand{\dec}[2]{\ensuremath{\overset{#1}{\underset{#2}{\gtrless}}}}
\newcommand{\myvec}[1]{\ensuremath{\begin{pmatrix}#1\end{pmatrix}}}
\newcommand{\mydet}[1]{\ensuremath{\begin{vmatrix}#1\end{vmatrix}}}
\numberwithin{equation}{subsection}
\makeatletter
\@addtoreset{figure}{problem}
\makeatother
\let\StandardTheFigure\thefigure
\let\vec\mathbf
\renewcommand{\thefigure}{\theproblem}
\def\putbox#1#2#3{\makebox[0in][l]{\makebox[#1][l]{}\raisebox{\baselineskip}[0in][0in]{\raisebox{#2}[0in][0in]{#3}}}}
     \def\rightbox#1{\makebox[0in][r]{#1}}
     \def\centbox#1{\makebox[0in]{#1}}
     \def\topbox#1{\raisebox{-\baselineskip}[0in][0in]{#1}}
     \def\midbox#1{\raisebox{-0.5\baselineskip}[0in][0in]{#1}}
\vspace{3cm}
\title{\textbf{Optimization Assignment - Advanced} }
\author{Dukkipati Vijay Sai}
\maketitle
\newpage
\bigskip
\renewcommand{\thefigure}{\theenumi}
\renewcommand{\thetable}{\theenumi}
Get Python code for the figure from 
\begin{lstlisting}
https://github.com/dukkipativijay/Fwciith2022/tree/main/Assignment%201/Codes/src
\end{lstlisting}
Get LaTex code from
\begin{lstlisting}
https://github.com/dukkipativijay/Fwciith2022/tree/main/Assignment%201%20-%20Assembly/Codes
\end{lstlisting}
%
\section{Question}
\centering
\textbf{\textit{Q(26), Class - 12, CBSE Paper, 2015}}\\
\vspace{0.25cm}
\raggedright
\textbf{The sum of the surface areas of a cuboid with sides x, 2x, $\frac{x}{3}$ and a sphere is given to be constant. Prove that the sum of their volumes is minimum, if x is equal to three times the radius of sphere. Also find the minimum value of the sum of their volumes.}\\
\raggedright
\section{Solution}

\vspace{0.25cm}
\raggedright
Let x, 2x, $\frac{x}{3}$ be the lengths of the sides of the Cuboid and r be the radius of the Sphere\\
\vspace{0.2cm}
Surface Area of the cuboid is,
\begin{align*}
A_1 = 2 (lb + bh + hl)\\
A_1 = 2x^2 + \frac{2x^2}{3} + \frac{x^2}{3}
\end{align*}
\begin{align}
A_1 = 6x^2
\label{eq1}
\end{align}
\raggedright
Surface Area of the Sphere is,
\begin{align}
A_2 = 4 \pi r^2
\label{eq2}
\end{align}
By the given problem from Eq. \eqref{eq1} and \eqref{eq2} we can write,
\begin{align}
6x^2 + 4 \pi r^2 = k \hspace{0.1cm}(constant)
\label{eq3}
\end{align}
Now, the volumes of the cuboid and the sphere can combined will be,
\begin{align}
V = \frac{2}{3} x^3 + \frac{4}{3} \pi r^3
\label{eq4}
\end{align}
Now putting the value of r from Eq. \eqref{eq3} in the Eq. \eqref{eq4} we get,
\begin{align}
V = \frac{2}{3} x^3 + \frac{4}{3} \Bigg( {\frac{k - 6x^2}{4 \pi}}\Bigg)^{\frac{3}{2}}
\label{eq5}
\end{align}
Given to prove that x = 3r for $V_{min}$. And to find the $V_{min}$\\
\vspace{0.25cm}
Hence, differentiating Eq. \eqref{eq5} w.r.t x we get,\\
\begin{align*}
\frac{dV}{dr} = \frac{d}{dr} \Bigg( \frac{2}{3} x^3 + \frac{4}{3} \Bigg( {\frac{k - 6x^2}{4 \pi}}\Bigg)^{\frac{3}{2}} \Bigg)
\end{align*}
By simplification we get,\\
\begin{align}
\frac{dV}{dr} = 2x^2 - 6x \hspace{0.1cm} \Bigg( {\frac{k - 6 x^2}{4 \pi}}\Bigg)^\frac{1}{2}
\label{eq6} 
\end{align}
Now Substitute k from Eq. \eqref{eq1} and then solving for x by equating the \eqref{eq6} to zero,\\
\begin{align}
x = 3r
\label{eq7}
\end{align}
Differentiating Eq. \eqref{eq5} once again w.r.t x and substituting \eqref{eq7} it is observed that,
\begin{align}
\frac{d^2V}{dr^2} > 0
\end{align}
Hence it is a point of Minima.\\
\vspace{0.2cm}
Therefore, the sum of their volumes is minimum, if x = 3r.\\
\vspace{0.25cm}
\centering
Hence Proved.\\
\vspace{0.25cm}
\raggedright
Now to find $V_{min}$, By Substituting Eq. \eqref{eq7} in Eq. \eqref{eq4} we get,\\
\begin{align*}
V_{min} = \frac{2}{3} ( 3r )^3 + \frac{4}{3} \pi r^3
\end{align*} 
\begin{align*}
\therefore V_{min} = r^3 \Big( 18 + \frac{4}{3} \pi \Big)
\end{align*}
\section{Verification Using Geomtric Programming}
\vspace{0.25cm}
\raggedright
Disciplined geometric programming, DGP is a subset of log-log-convex program (LLCP). An LLCP is defined as,\\
\vspace{0.5cm}
\hspace{2cm} minimize $f_0(x)$\\
\vspace{0.2cm}
\hspace{2cm} subject to $f_i(x) \leq \tilde{f}_i$, i = 1,2,...,m.\\
\vspace{0.2cm}
\hspace{3.8cm} $g_i(x) = \tilde{g}_i$, i = 1,2,...,p.
\begin{align}
\label{eq8}
\end{align}

where the functions $f_i$
				are log-log convex, $\tilde{f}_i$
 are log-log concave, and the functions $g_i$
				and $\tilde{g}_i$
 are log-log affine. An optimization problem with constraints of the above form in which the goal is to maximize or minimize a log-log concave function is also an LLCP. These LLCPs generalize geometric programming.\\

\vspace{0.25cm}

The given problem can be formulated as a DGP as,
\begin{align}
V = \min\limits_{r,h} \hspace{0.2cm} \frac{2}{3} x^3 + \frac{4}{3} \pi r^3\\
s.t \hspace{0.5cm} 6 x^2 + 4 \pi r^2 = k \hspace{0.2cm} (constant)
\label{eq9}
\end{align}
By assuming k = 40 as input, and solving the above DGP Equations using Cvxpy we get,

\begin{align}
V_{min} = 136.11\\
x = 3.92\\
r = 1.30
\end{align} 

Hence it is proved that $ x = 3r $ for $V_{min}$.\\
\vspace{0.5cm}

\end{document}
Footer